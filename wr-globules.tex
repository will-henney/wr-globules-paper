\RequirePackage{amsmath}
\documentclass[twocolumn, times]{aastex631}
\usepackage[spanish,es-minimal,english]{babel}
\usepackage[utf8]{inputenc}
\usepackage{natbib}
%\usepackage{microtype}
\usepackage{hyperref}
\usepackage{savesym}
\savesymbol{tablenum}
\usepackage{siunitx}
\restoresymbol{SIX}{tablenum}
\usepackage[varg]{newtxmath}
\usepackage{newtxtext}
\usepackage{booktabs}
\usepackage{array}   % for \newcolumntype macro
\newcolumntype{L}{>{$}l<{$}} % math-mode version of lrc column types
\newcolumntype{R}{>{$}r<{$}} 
\newcolumntype{C}{>{$}c<{$}} 

\bibliographystyle{aasjournal}

\newcommand\ION[2]{#1\,\scalebox{0.9}[0.8]{\uppercase{#2}}}
\newcounter{ionstage}
\renewcommand{\ion}[2]{\setcounter{ionstage}{#2}% 
  \ensuremath{\mathrm{#1\,\scriptstyle\Roman{ionstage}}}}
\newcommand\hii{\ion{H}{2}}
\newcommand\hi{\ion{H}{1}}
\newcommand\heii{\ion{He}{2}}
\newcommand\hei{\ion{He}{1}}
\newcommand\siii{[\ion{S}{3}]}
\newcommand\sii{[\ion{S}{2}]}
\newcommand\oiii{[\ion{O}{3}]}
\newcommand\ariii{[\ion{Ar}{3}]}
\newcommand\ariv{[\ion{Ar}{4}]}
\newcommand\ARIV{[\ION{Ar}{iv}]}
\newcommand\Raman{\ensuremath{_{\text{Raman}}}}
\def\th#1#2{\(\theta^{#1}\)\,Ori~#2}
\newcommand\wn{\ensuremath{\tilde{\nu}}}
\newcommand\Wav[1]{\ensuremath{\lambda #1}}

% Chemical formulae
\newcommand*\chem[1]{\ensuremath{\mathrm{#1}}}
% Atomic term symbols
\newcommand\Config[1]{\ensuremath{\mathrm{#1}}}
\newcommand\Term[3]{\ensuremath{\mathrm{#1\ ^{#2}#3}}}
\newcommand\Level[4]{\ensuremath{\mathrm{#1\ ^{#2}#3_{#4}}}}

\newcommand\ha{\ensuremath{\text{H}\alpha}}
\newcommand\hb{\ensuremath{\text{H}\beta}}
\newcommand\lya{\ensuremath{\text{Ly}\alpha}}
\newcommand\lyb{\ensuremath{\text{Ly}\beta}}

\newcommand{\wind}{\ensuremath{_{\text{w}}}}
\newcommand\snrj{SNR~J\num{0059.4}\num{-7210}}

\begin{document}
\title{Shielded from the wind: evaporating neutral globules around a Wolf-Rayet star}
\shorttitle{Evaporating globules around WR~124}
\author[0000-0001-6208-9109]{William J. Henney}
\affiliation{%
  \foreignlanguage{spanish}{Instituto de Radioastronomía y
    Astrofísica, Universidad Nacional Autónoma de México, Apartado
    Postal 3-72, 58090 Morelia, Michaoacán, Mexico}
}
\email{w.henney@irya.unam.mx}
\correspondingauthor{William J. Henney}

\author[0000-0000-0000-0000]{Roberto Reyes-Cabañas}
\affiliation{%
  \foreignlanguage{spanish}{Instituto de Radioastronomía y
    Astrofísica, Universidad Nacional Autónoma de México, Apartado
    Postal 3-72, 58090 Morelia, Michaoacán, Mexico}
}

\author[0000-0002-5456-4472]{S. Jane Arthur}
\affiliation{%
  \foreignlanguage{spanish}{Instituto de Radioastronomía y
    Astrofísica, Universidad Nacional Autónoma de México, Apartado
    Postal 3-72, 58090 Morelia, Michaoacán, Mexico}
}

% \author[0000-0002-5406-0813]{Jesús A. Toalá}
% \affiliation{%
%   \foreignlanguage{spanish}{Instituto de Radioastronomía y
%     Astrofísica, Universidad Nacional Autónoma de México, Apartado
%     Postal 3-72, 58090 Morelia, Michaoacán, Mexico}
%   }

\begin{abstract}
  The circumstellar nebula M1-67 around the Wolf-Rayet star WR~124
  contains hundreds of small neutral globules, as revealed by JWST images.
  However, despite the powerful stellar wind from the star, the globules
  do not interact directly with the wind.
  Instead, they are hydrodynamically shielded by a transonic warm ionized
  flow away from their surfaces that is induced by the Lyman continuum radiation
  of the star.
  This inwardly directed photoevaporation flow shocks against the
  outflowing stellar wind to form dense hemispherical ionized shells
  that are a few times larger than the globules
  and which contribute a significant fraction of the recombination line luminosity
  of the nebula.
  We analyze archival HST images of the nebula and estimate the size, brightness,
  and ionized density of each photoevaporation flow and accompanying interaction shell.
  We show that \textit{something, something, something, \dots}.
  We compare the globule-wind interaction around WR~124 with similar interactions
  seen elsewhere, such as photoevaporating disks (proplyds) in the Orion Nebula.
\end{abstract}

\keywords{Circumstellar matter; Stars: winds, outflows}

%\object{M42}

\section{Introduction}
\label{sec:introduction}

\section{Conclusions}
\label{sec:conclusions}

\begin{acknowledgments}
  Thank you.
\end{acknowledgments}

\facilities{HST; JWST}

\bibliography{wr-globules-refs}

\end{document}

%%% Local Variables:
%%% mode: latex
%%% TeX-master: t
%%% End:
